\begin{recipe}[label=gnocchi]{Gnocchi di patate}
    \begin{header}
        \portion{1}[porzione]
        \source{Davide Ponzini}
        \recipedate{\tilde 2022}

        \preparationTime{\timeM{20}}
        \cookingTime{\timeM{45}}{Fiamma alta}[Bollitura]
        \cookingTime{\timeM{2}}{Fiamma alta}[Cottura]
    \end{header}
    
    \begin{ingredients}[1]
        \ingredient[200][g]{Patate}
        \ingredient[60][g]{Farina '0}
        \ingredient[1][g]{Sale}
    \end{ingredients}
    
    \begin{preparation}
        \step Lessare le patate con la buccia.
        \step Sbucciare le patate, schiacciarle.
        \step Unire farina ed impastare.
        \step Stendere l'impasto, ricavare filoncini e strapparli a 2 cm di lunghezza.
        \step Rigare passando sui lembi di una forchetta, per dare la tipica forma.
        \step Cuocere.
    \end{preparation}
    
    \begin{suggestion}
        \suggestionMark \textbf{Non usare patate novelle}: non sono adatte a diventare gnocchi e si sfaldano durante la cottura.
        \suggestionMark Non lavorare eccessivamente l'impasto, se no gli gnocchi diventeranno duri durante la cottura.
        \suggestionMark Girare la forchetta al contrario, poggiarla sulla spianatoia e passare gli gnocchi dalla punta verso il manico.
    \end{suggestion}
\end{recipe}
