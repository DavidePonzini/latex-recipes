\begin{recipe}{Lasagne}
    \begin{header}
        \portion{4+}
        \source{Abbate Maria Luisa}
        \recipedate{\tilde 2021}
    
        \preparationTime{\timeM{30}}[esclusa preparazione \linkRecipe{ragu}[ragù] e \linkRecipe{besciamella}]
        \bakingTimeTopbottom{\timeH{1}}{180}
    \end{header}
    
    \begin{ingredients}
        \ingredient[500][g carne]{Ragù}
        \ingredient[1][l latte]{Besciamella}
        \ingredient[??]{Sfoglia lasagne}
        \ingredient{Formaggio grattugiato}
        \ingredient{Olio}
    \end{ingredients}
    
    \begin{preparation}
        \step Mettere un velo d'olio su un testo. Coprire bene i bordi ed il centro, disegnando una serpentina.
        \step Versare uno strato di ragù e besciamella.
        \step Posare uno strato di lasagne (sovrapposte per circa 1 cm).
        \step Aggiungere strato di ragù e besciamella.
        \step Aggiungere strato di prosciutto e formaggio.
        \step Ripetere da \textcolor{red}{3} fino a fine ingredienti.
        \step Posare un ultimo strato di lasagne.
        \step Coprire il tutto con uno strato di ragù e besciamella.
        \step Cuocere in forno per \timeM{30}.
        \step Controllare la cottura.
        \step Cuocere in forno per altri \timeM{20-30}.
    \end{preparation}
    
    \begin{suggestion}
        \suggestionMark Fare attenzione ai bordi, si rischia di bruciare.
    \end{suggestion}
\end{recipe}
