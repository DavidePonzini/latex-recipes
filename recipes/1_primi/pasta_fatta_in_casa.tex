\begin{recipe}{Pasta fatta in casa}
    \begin{header}
        \portion{`$Farina * 1.55$'}[g]
        \source{Internet}
        \recipedate{\tilde 2020}
    
        \preparationTime{\timeH{2}}
        \cookingTime{\timeM{4}}{Fiamma alta}
    \end{header}
    
    \begin{introduction}
        Unire alla farina o le uova o l'acqua.
    \end{introduction}
    
    \begin{ingredients}
        \ingredient[100][g]{Farina}
        \ingredient[55][g]{Acqua}
        \ingredient[1]{Uovo}
    \end{ingredients}
    
    \begin{preparation}
        \step Impastare farina con acqua (o uova).
        \step Lasciare riposare l'impasto per \timeM{30}, coperto da pellicola.
        \step Prendere l'impasto parte per parte, infarinarlo ed stenderlo con la macchina per la pasta.
        \step Infarinare ulteriormente ed usare la macchina per la pasta per tagliarla.
        \step Posare attentamente la pasta pronta su carta forno, assicurandosi non si sovrapponga.
        \step Mettere tutta la pasta in congelatore.
    \end{preparation}
    
    \begin{suggestion}
        \suggestionMark Usare farina di semola quando si deve stendere l'impasto, in quanto molto più comoda da lavorare.
        
        \suggestionMark Non stendere l'impasto troppo velocemente, o si rovinerà.
            Se si rovina, basta reimpastarlo da capo.
    \end{suggestion}
\end{recipe}
