\begin{recipe}[label=insalata_di_riso]{Insalata di riso}
    \begin{header}
        \portion{6}
        \source{Internet (modificata)}
        \recipedate{Luglio 2023}
    
        \preparationTime{\timeH{2}}
        \cookingTime{\timeM{3}}{Fiamma alta}[Piselli]
        \cookingTime{\timeM{15}}{Fiamma media}[Riso]
        \cookingTime{\timeM{8}}{Fiamma alta}[Uova]
        \preparationWait{\timeH{2+}}[Frigorifero]
    \end{header}
    
    \begin{ingredients}[10]
        \ingredient[300][g]{Riso Arborio}
        \ingredient[150][g]{Pomodorini}
        \ingredient[100][g]{Prosciutto cotto a cubetti}
        \ingredient[75][g]{Peperoni rossi}
        \ingredient[75][g]{Peperoni gialli}
        \ingredient[80][g]{Olive denocciolate}
        \ingredient[200][g]{Tonno sott'olio sgocciolato}
        \ingredient[150][g]{Formaggio}
        \ingredient[80][g]{Piselli}
        \ingredient[3]{Uova sode}
        \ingredient[1][scatoletta]{Mais}
        \ingredient[200][g]{Maionese \textit{(tonnata)}}
        \ingredient{Sale}
    \end{ingredients}
    
    \begin{preparation}
        \step Far bollire l'acqua.
        \step Far cuocere i piselli per \timeM{3}.
        \step Scolare i piselli e lasciarli raffreddare.
        \step Cuocere il riso \underline{al dente}.
        \step Scolare il riso e lasciarlo raffreddare.
        \step Preparare le \linkRecipe{uova_sode}[uova sode].
        \step Tagliare verdure e formaggio a cubetti, metterli in una ciotola grande.
        \step Una volta raffreddati, unire i piselli.
        \step Aggiungere maionese e mescolare.
        \step Aggiungere il riso e mescolare.
        \step Aggiustare di sale se necessario.
        \step Aggiungere le uova sode tagliate a dischetti, mescolare delicatamente.
        \step Far riposare in frigorifero.
    \end{preparation}
\end{recipe}