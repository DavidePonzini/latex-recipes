\begin{recipe}{Pane}
    \begin{header}
        \portion{\tilde 4}[panini]
        \source{Davide Ponzini}
        \recipedate{2020}

        \preparationTime     {\timeM{30}}
        \preparationWait     {\timeH{2-4}}
        \bakingTimeFan       {\tilde \timeM{20}}{220}
    \end{header}
    
    \begin{introduction}
        Conservazione: tenere al riparo dall'aria e dalla luce.
    \end{introduction}
    
    \begin{ingredients}
        \ingredient[300][g]{Farina '0}
        \ingredient[200][g]{Farina '1 e/o '2}
        \ingredient[20][g]{Olio d'oliva}
        \ingredient[280][g]{Acqua}
        \ingredient[6][g]{Lievito fresco}
        \ingredient[6][g]{Sale fino}

        \ingredientSection{Extra (opzionale)}
        \ingredient{Semi di girasole}
        \ingredient{Semi di papavero}
        \ingredient{Sesamo}
    \end{ingredients}
    
    \begin{preparation}
        \step Preparare impasto.
        \step Lasciare riposare fino a raddoppio.
        \step Dividere in panetti, dare forma.
        \step Lasciare riposare fino a raddoppio.
        \step Cuocere in forno.
    \end{preparation}

    \begin{suggestion}[2cm]
        \suggestionMark Tenere in forno un pentolino con acqua, per evitare che il pane diventi mollo col tempo.
    \end{suggestion}

    \begin{hint}
        Ricetta perfezionata durante il COVID-19.
    \end{hint}
\end{recipe}
