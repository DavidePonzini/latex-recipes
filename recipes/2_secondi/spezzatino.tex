\begin{recipe}{Spezzatino}
    \begin{header}
        \portion{4}
        \source{Paolo Orunesu}
        \recipedate{\tilde 2019}
    
        \preparationTime{\timeM{15}}
        \cookingTime{\timeH{4}}{Fiamma medio-bassa}
        \preparationWait{\tilde \timeH{8}}
    \end{header}
    
    \begin{ingredients}
        \ingredient[1][kg]{Spezzatino}
        \ingredient{Sale}
        \ingredient{Olio}
        \ingredient[4]{Carote}
        \ingredient[1]{Cipolla}
        \ingredient[1]{Spicchi aglio}
        \ingredient[6]{Foglie alloro}
        \ingredient[2]{Foglie basilico}
        \ingredient[2]{Foglie menta}
        \ingredient[1]{Rametto rosmarino}
        \ingredient[50][g]{Piselli}
        \ingredient{Olive}
        \ingredient[poco]{Finocchio selvatico}
        \ingredient[2][bicchieri]{Vino rosso corposo}
        \ingredient{\textit{(altre erbe)}}
    \end{ingredients}
    
    \begin{preparation}
        \step Mettere in una pentola olio, carne, cipolla, aglio e carote.
        \step Far cuocere fino a che la carne cambia colore (\timeM{15-20}).
        \step Aggiungere vino, sale erbe (tranne finocchio).
        \step Far cuocere fino a che carne è molto morbida.
        \step Aggiungere piselli, olive, finocchio selvatico.
        \step Far cuocere per \timeM{30}.
        \step Far riposare per qualche ora.
    \end{preparation}
    
    \begin{suggestion}
        \suggestionMark In caso di necessità, aggiungere acqua.
    \end{suggestion}
    
\end{recipe}
