\begin{recipe}{Focaccia}
    \begin{header}
        \portion{1}[teglia (\textit{500 g)}]
        \source{Daniele Traversaro}
        \recipedate{Giugno 2023}

        \preparationTime    {\timeM{30}}
        \preparationWait    {\timeH{2-4}}
        \bakingTimeFan      {\timeM{15-20}}{180}
    \end{header}
    
    \begin{ingredients}[13]
        \ingredientSection{Impasto}
        \ingredient[400][g]{Farina}
        \ingredient[250][g]{Acqua \textit{(tiepida)}}
        \ingredient[9][g]{Sale}
        \ingredient[25][g]{Olio EVO}
        \ingredient[12][g]{Lievito fresco}
        \ingredient[1][cucchiaino]{Zucchero}

        \ingredientSection{Salamoia}
        \ingredient[150][g]{Acqua}
        \ingredient[25][g]{Olio EVO}
        \ingredient{Sale}
    \end{ingredients}
    
    \begin{preparation}
        \step Prepare impasto.
        \step Lasciarlo riposare per \timeM{15} coperto da un canovaccio umido.
        \step Ripiegare impasto su se stesso un paio di volte.
        \step Ungere la teglia.
        \step Mettere l'impasto sulla teglia, lasciarlo lievitare in forno spento per \timeM{60}.
        \step \textit{Controllare che la teglia sia ancora adeguatamente unta}.
        \step Schiacciare l'impasto per coprire tutta la teglia.
        \step Lasciare lievitare in forno per \timeM{30}.
        \step Fare i buchi con le dita, cospargere con tutta la salamoia.
        \step Lasciare lievitare in forno per \timeM{60}.
        \step Cuocere in forno preriscaldato.
    \end{preparation}
\end{recipe}
