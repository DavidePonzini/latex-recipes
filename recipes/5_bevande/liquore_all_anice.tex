\begin{recipe}{Liquore all'anice}
    \begin{header}
        \portion{4+}[litri]
        \source{Internet}
        \recipedate{\tilde 2019}
    
        \preparationTime{\timeM{10}}[Ingredienti]
        \preparationWait{\timeD{30}}
        \preparationTime{\timeM{15}}[Sciroppo]
        \preparationWait{\timeH{4}}[Far raffreddare lo sciroppo]
        \preparationWait{\timeD{60}}
    \end{header}

    \begin{ingredients}
        \ingredient[2][L]{Alcohol per liquori}
        \ingredient[2][L]{Acqua}
        \ingredient[1.5][kg]{Zucchero}
        \ingredient[150][g]{Anice stellato}
        \ingredient[15]{Chiodi di garofano}
        \ingredient[5]{Stecche di cannella}
    \end{ingredients}
    
    \begin{preparation}
        \step Mettere in una bottiglia alcohol, anice, chiodi di garofano, cannella.
        \step Lasciare riposare, agitando ogni giorno.
        \step Preparare sciroppo con acqua e zucchero.
        \step Far bollire per \timeM{2}.
        \step Far raffreddare sciroppo.
        \step Unire al liquore.
        \step Lasciare riposare.
    \end{preparation}
    
    \begin{suggestion}
        \suggestionMark Il liquore migliora di sapore dopo un po' di tempo. Si consiglia di farlo riposare ben oltre il tempo indicato.
        
        \suggestionMark Costo stimato di produzione: \tilde 10.37€/litro\footnote{
            Marzo 2023.
            Consultare codice sorgente commentato per informazioni dettagliate.
            % stecche cannella       2.36 €
            % chiodi di garofano     0.14 €
            % anice                ~ 3.00 € (comprato da Jawad, in Italia anche 60€/kg)
            % alcohol               33.80 €
            % zucchero               2.17 €
            % acqua                    -- €
            % _______________________________________
            % totale                41.47 € per 4 lt
            %  
            % escluse dal conteggio:
            %       bottiglie        4.29 € (aranciata Lurisia, 0.275 lt/bottiglia)
        }.
    \end{suggestion}
\end{recipe}
