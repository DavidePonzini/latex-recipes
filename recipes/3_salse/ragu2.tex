\begin{recipe}[label=ragu]{Ragù 2.0}
    \begin{header}
        \portion{\tilde 12}[barattoli da 500 ml]
        \source{Davide Ponzini}
        \recipedate{\tilde Marzo 2023}
    
        \preparationTime{\timeM{45}}
        \cookingTime{\timeH{3-4}}{Fiamma bassa}
    \end{header}
    
    \begin{ingredients}[10]
        \ingredient{Olio}
        \ingredient[125][g]{Burro}
        \ingredientSection{Verdure}
        \ingredient[40][g]{Aglio}
        \ingredient[250][g]{Cipolla}
        \ingredient[750][g]{Carote}
        \ingredient[250][g]{Sedano}
        \ingredientSection{Carne}
        \ingredient[1000][g]{Tritato bovino}
        \ingredient[1000][g]{Tritato bovino}
        \ingredient[500][g]{Mortadella tritata}
        \ingredient{Sale}
        \ingredientSection{Altro}
        \ingredient[5.5][g]{Pepe nero}
        \ingredient[5][bottiglie]{Salsa di pomodoro}
        \ingredient[1\sfrac{1}{4}][bottiglie]{Vino rosso corposo}
    \end{ingredients}
    
    \begin{preparation}
        \step Sciogliere burro ed olio in pentola.
        \step Tritare verdure.
        \step Cuocere verdure fino a che non diventano morbide (\timeM{\tilde{}10}).
        \step Aggiungere tritato e pepe.
        \step Cuocere fino a che carne non cambia bene colore (\timeM{\tilde{}10}).
        \step Aggiungere vino.
        \step Cuocere fino a che non si è un po' asciugato (\timeM{\tilde{}15}).
        \step Aggiungere salsa.
        \step Cuocere per \timeH{3-4}.
    \end{preparation}
    
    \begin{suggestion}[1cm]
        \suggestionMark Usare pentola con fondo antiaderente per non bruciare il fondo.
        \suggestionMark Non aggiungere olio dopo la carne, brucia il fondo in ogni caso.
    \end{suggestion}
    
    \begin{hint}
        Ricetta migliorata grazie ai proff. del Marco Polo.
    \end{hint}
    
\end{recipe}
