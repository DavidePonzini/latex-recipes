\begin{recipe}{Fonduta di formaggio}
    \begin{header}
        \portion{100 g}[per persona]
        \source{Manuela Del Bianco}
        \recipedate{\tilde 2017}
    
        \preparationTime{\timeM{10}}
        \cookingTime{\timeH{1+}}{Fiamma bassa}
    \end{header}
    
    \begin{introduction}
        Prestare particolare attenzione alla scelta dei formaggi:
        
        \begin{itemize}
            \item Scamorza ok solo senza buccia
            \item Formaggi duri no (non si sciolgono bene)
                \begin{itemize}
                    \item Grana/parmigiano
                    \item Pecorino
                \end{itemize}
        \end{itemize}
    \end{introduction}
    
    \begin{ingredients}
        \ingredient[\sfrac{1}{3}]{Toma piemontese}
        \ingredient[\sfrac{1}{3}]{Raschera}
        \ingredient[\sfrac{1}{3}]{Fontina}
        \ingredient{Latte}
        \ingredient{Noce moscata}
    \end{ingredients}
    
    \begin{preparation}
        \step Tagliare formaggi a cubetti.
        \step Metterli in un pentolino con il latte.
        \step Cuocere a fuoco basso-medio fino a che il latte non bolle.
        \step Aggiungere noce moscata.
        \step Mescolare costantemente per \timeH{1}.
    \end{preparation}
    
    \begin{suggestion}
        \suggestionMark Per far rassodare più velocemente, è possibile aggiungere della farina alla fine.
    \end{suggestion}
\end{recipe}
