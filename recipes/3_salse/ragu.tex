\begin{recipe}{Ragù}
    \begin{header}
        \portion{\tilde 10}[barattoli da 500 ml]
        \source{Samuele Daga \textit{(modificata)}}
        \recipedate{\tilde 2016}
    
        \preparationTime{\timeM{45}}
        \cookingTime{\timeH{3-4}}{Fiamma bassa}
    \end{header}
    
    \begin{ingredients}[10]
        \ingredient{Olio}
        \ingredient[50][g]{Burro}
        \ingredient[15][g]{Aglio}
        \ingredient[100][g]{Cipolla}
        \ingredient[300][g]{Carote}
        \ingredient[100][g]{Sedano}
        \ingredient[1][kg]{Tritato bovino}
        \ingredient{Pepe}
        \ingredient{Sale}
        \ingredient[4][bottiglie]{Salsa di pomodoro}
        \ingredient[2][bicchieri]{Vino rosso corposo}
    \end{ingredients}
    
    \begin{preparation}
        \step Sciogliere burro ed olio in pentola.
        \step Tritare verdure.
        \step Cuocere verdure fino a che non diventano morbide (\timeM{\tilde{}10}).
        \step Aggiungere tritato e pepe.
        \step Cuocere fino a che carne non cambia bene colore (\timeM{\tilde{}10}).
        \step Aggiungere vino.
        \step Cuocere fino a che non si è un po' asciugato (\timeM{\tilde{}15}).
        \step Aggiungere salsa.
        \step Cuocere per \timeH{3-4}.
    \end{preparation}
    
    \begin{suggestion}
        \suggestionMark Usare pentola con fondo antiaderente per non bruciare il fondo.
        \suggestionMark Non aggiungere olio dopo la carne, brucia il fondo in ogni caso.
    \end{suggestion}
    
    \begin{hint}
        Grazie Sammy per avermi fatto iniziare a cucinare!
    \end{hint}
\end{recipe}
